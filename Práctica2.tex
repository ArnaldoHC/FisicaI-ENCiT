\documentclass[journal]{IEEEtran}
\usepackage[spanish]{babel}
\usepackage[utf8]{inputenc}
\usepackage{amsmath}
\usepackage{graphicx}
\usepackage[colorinlistoftodos]{todonotes}
\usepackage{float}
\usepackage{listings}
\markboth{Universidad Nacional Autónoma de México}{Universidad Nacional Autónoma de México}
\usepackage{cite}

\title{Título de la Práctica}
\markboth{Universidad Nacional Autónoma de México} 

\author{
\authorblockA{Nombre 1 \\Nombre 2 \\...\\ Docente: ... \\ 
\today  }}





\begin{document}
\maketitle

\begin{abstract}
En este informe se presenta detalladamente el desarrollo de ...\\

Palabras Claves: Péndulo, velocidad máxima ... .
\end{abstract}


\section{Introducción}
 El péndulo se describe como...

 ...a partir de las ecuaciones 
$$v_{max}=\sqrt{2gL(1-\text{cos}_A)}$$
 


\section{Objetivos}

\subsection{Objetivo General}
Obtener la velocidad máxima $(v_max)$ de un péndulo por medio del ...

\subsection{Objetivos específicos}

\begin{enumerate}
    \item Demostrar que la velocidad no depende de la masa del objeto. 
    \item Demostrar que la velocidad depende de la longitud del hilo.
    \item Demostrar que la velocidad depende del ángulo.
\end{enumerate}


\section{Materiales}

\begin{enumerate}
    \item Hilo de trenado de naylon de calibre \#.
    \item Botella (Objeto de péndulo).
    \item Lapicero (Objeto de péndulo).
    \item ... (Objeto de péndulo).
\end{enumerate}

\section{Procedimiento}

\subsection{Armado experimental}

Se eligió como punto de pivote un...

 \begin{figure}[H]
    \centering
    \includegraphics[width=8cm, height=5cm]{figura de montaje.jpg}
    \caption{Figura 1. Montaje del péndulo.}
\end{figure}

 
 
 \subsection{Toma de video}
El video se tomó on una camara de celular con una resolución de ... y una velocidad de camptura de ...cps ...

 \begin{figure}[H]
    \centering
    \includegraphics[width=8cm, height=6cm]{figura de toma de video.jpeg}
    \caption{Medición de la velocidad por medio de los fotogramas.}
\end{figure}
 
 

\section{Resultados}

\begin{table}[H]
\centering
\caption{Tabla de diferentes masas...}
\begin{tabular}{c|c|c|c|c|}
\hline 
Lanzamiento&$L\;[m]$ & $theta_i\;[^{\circ}]$ & $v_i(R,\theta)\;[m/s]$ & $v_i\;[m/s] del video$ \\ 

\hline 
1&100.2 & 45 & 36.8&37.3 \\ 
\hline 
2& &  &  & \\ 
\hline 
3& &  &  & \\ 
\hline 
4& & & \\ 
\hline 
...
\end{tabular} 
\end{table}

\begin{table}[H]
\centering
\caption{Tabla de diferentes longitudes...}
\begin{tabular}{c|c|c|c|c|}
\hline 
Lanzamiento&$L\;[m]$ & $theta_i\;[^{\circ}]$ & $v_i(R,\theta)\;[m/s]$ & $v_i\;[m/s] del video$ \\ 

\hline 
1&100.2 & 45 & 36.8&37.3 \\ 
\hline 
2& &  &  & \\ 
\hline 
3& &  &  & \\ 
\hline 
4& & & \\ 
\hline 
...
\end{tabular} 
\end{table}

\begin{table}[H]
\centering
\caption{Tabla de diferentes ángulos...}
\begin{tabular}{c|c|c|c|c|}
\hline 
Lanzamiento&$L\;[m]$ & $theta_i\;[^{\circ}]$ & $v_i(R,\theta)\;[m/s]$ & $v_i\;[m/s] del video$ \\ 

\hline 
1&100.2 & 45 & 36.8&37.3 \\ 
\hline 
2& &  &  & \\ 
\hline 
3& &  &  & \\ 
\hline 
4& & & \\ 
\hline 
...
\end{tabular} 
\end{table}
\section{Cuestionario}

\begin{enumerate}
    \item A partir de los resultados de variar la masa del péndulo, ¿porque la velocidad no depende de la masa? Explica ampliamente tu respuesta.
    
    \item ¿Cuál es el comportamiento del péndulo cuando se incrementa la longitud del hilo? 
    
    \item ¿Porque para ángulos grandes el péndulo no se comporta como la teoría descrita en clase?
    
    \item ¿Cuales son las aproximaciones que se deben considerar en un péndulo para que la ecuación analizada en esta práctica sea cierta? Explica ampliamente tu respuesta.
    
   
    
\end{enumerate}




 \bibliographystyle{elsarticle-harv} 
 \bibliography{references}




\end{document}




