\documentclass[journal]{IEEEtran}
\usepackage[spanish]{babel}
\usepackage[utf8]{inputenc}
\usepackage{amsmath}
\usepackage{graphicx}
\usepackage[colorinlistoftodos]{todonotes}
\usepackage{float}
\usepackage{listings}
\markboth{Universidad Nacional Autónoma de México}{Universidad Nacional Autónoma de México}
\usepackage{cite}

\title{Péndulo}
\markboth{Universidad Nacional Autónoma de México} 

\author{
\authorblockA{Nombre 1 \\Nombre 2 \\...\\ Docente: ... \\ 
\today  }}





\begin{document}
\maketitle

\begin{abstract}
En este informe se presenta detalladamente el desarrollo de ...\\

Palabras Claves: Tiro parabólico, ... .
\end{abstract}


\section{Introducción}
 El tiro parabólico ...

 a partir de las ecuaciones 
$$v_i=$$
 


\section{Objetivos}

\subsection{Objetivo General}
Obtener la velocidad inicial $(v_i)$ de un...

\subsection{Objetivos específicos}

\begin{enumerate}
    \item Calcular el alcance $R$. 
    \item El ángulo de salida $theta_i$.
    \item ...
\end{enumerate}


\section{Materiales}

\begin{enumerate}
    \item Cañón de manufactura casera.
    \item Hoja cuadriculada.
    \item ...
\end{enumerate}

\section{Procedimiento}

\subsection{Armado experimental}

Se colocó el cañon...

 \begin{figure}[H]
    \centering
    \includegraphics[width=8cm, height=5cm]{figura de montaje.jpg}
    \caption{Figura 1. Montaje del cañon.}
\end{figure}

 
 
 \subsection{Toma de video}
El video se tomó on una camara de celular con una resolución de ... y una velocidad de camptura de ...cps ...

 \begin{figure}[H]
    \centering
    \includegraphics[width=8cm, height=6cm]{figura de toma de video.jpeg}
    \caption{medicion de los amperios con un resistor de $33K\Omega$.}
\end{figure}
 
 
 \subsection{Circuito Serie Con Resistor De $47k\Omega$}
\section{Resultados}

\begin{table}[H]
\centering
\caption{Tabla de Valores...}
\begin{tabular}{c|c|c|c|c|}
\hline 
Lanzamiento&$R\;[m]$ & $theta_i\;[^{\circ}]$ & $v_i(R,\theta)\;[m/s]$ & $v_i\;[m/s] del video$ \\ 
\hline 
1&100.2 & 45 & 36.8&37.3 \\ 
\hline 
2& &  &  & \\ 
\hline 
3& &  &  & \\ 
\hline 
4& & & \\ 
\hline 
...
\end{tabular} 
\end{table}



\section{Cuestionario}

\begin{enumerate}
    \item ¿Cuál consideras es la mejor manera de medir el ángulo de salida?
    
    \item ¿Cuál es la influencia de la forma del proyectil en el cálculo de la velocidad inicial?
    
    \item ¿Que problemas presenta el elemento del armado experimental que controla el alcance del proyectil?
    
    \item ¿Cuáles son los problemas que encontraste al medir y alcular la velocidad inicial?
    
   
    
\end{enumerate}




 \bibliographystyle{elsarticle-harv} 
 \bibliography{references}




\end{document}




